\documentclass[a4paper]{ltjsarticle}

\usepackage{amsmath} % 数式を記述するためのパッケージ
\usepackage{graphicx} % 画像を挿入するためのパッケージ
\usepackage{url} % URLを記述するためのパッケージ
\usepackage{here} % 図表をその場所に表示するためのパッケージ
\usepackage{luacode} % ソースコードを記述するためのパッケージ
\usepackage{titling} % タイトルをカスタマイズするためのパッケージ
\usepackage{fancyhdr} % ヘッダーをカスタマイズするためのパッケージ
\usepackage{siunitx}
\usepackage{multirow}
\usepackage{bigdelim}
\usepackage{amssymb}
\usepackage{listings}
\usepackage{xcolor}
\usepackage{multicol}
\usepackage[subrefformat=parens]{subcaption}

\definecolor{codegreen}{rgb}{0,0.6,0}
\definecolor{codegray}{rgb}{0.5,0.5,0.5}
\definecolor{codepurple}{rgb}{0.58,0,0.82}
\definecolor{backcolour}{rgb}{0.95,0.95,0.92}

\lstdefinestyle{mystyle}{
    backgroundcolor=\color{backcolour},   
    commentstyle=\color{codegreen},
    keywordstyle=\color{blue},
    numberstyle=\tiny\color{codegray},
    stringstyle=\color{codepurple},
    basicstyle=\ttfamily\footnotesize,
    breakatwhitespace=false,         
    breaklines=true,                 
    captionpos=b,                    
    keepspaces=true,                 
    numbers=left,                    
    numbersep=5pt,                  
    showspaces=false,                
    showstringspaces=false,
    showtabs=false,                  
    tabsize=2,
    title=\lstname
}
\lstset{style=mystyle}


\preauthor{\begin{flushright}} % authorを右寄せにする
\postauthor{\end{flushright}}
\predate{\begin{flushright}} % dateを右寄せにする
\postdate{\end{flushright}}

\pagestyle{fancy}
\lhead{電気電子情報実験・演習第一 I3実験レポート}
\rhead{03240403 井上聡士}
\cfoot{\thepage}
\renewcommand{\headrulewidth}{0pt}


\title{電気電子情報実験・演習第一 I3実験レポート}
\author{電子情報工学科 \ 03240403 井上聡士}
\date{2024年6月28日}


\begin{document}
\maketitle % タイトルページを表示する
\section{Section}
\begin{align}
    V_1 &= \left( sL_1 + \frac{1}{sC_1} \right) I_1 - MsI_2 \\
    MsI_1 &= \left( sL_2 + \frac{1}{sC_2} + R \right) I_2 \\
    \therefore V_1 &= \left( sL_1 + \frac{1}{sC_1} - \frac{M^2s^2}{sL_2 + \frac{1}{sC_2} + R} \right) I_1
\end{align}
今、$\omega_0 = \frac{1}{\sqrt{L_1C_1}} = \frac{1}{\sqrt{L_2C_2}}$として、
\begin{align}
    V_1(t) &= V_e(t)\sin(\omega_0 t) \\
    I_1(t) &= I_e(t)\sin(\omega_0 t)
\end{align}
とすると、
\begin{align}
    V_1(s) &= V_e(s) * \frac{\omega_0}{s^2 + \omega_0^2} \\
    I_1(s) &= I_e(s) * \frac{\omega_0}{s^2 + \omega_0^2}
\end{align}
となる。また、
\begin{align}
    f(t)g(t) &= \int_{0}^{t}f(\tau)g(t-\tau)d\tau
\end{align}



\end{document}